This simulation system can simulate O\-N\-E robot soccer player for Robo\-Cup Middle Size League. It can be adapted for other purposes. Note that this package is only designed for demonstration. If you want to test multi-\/robot cooperation strategies, please refer to another repository\-: \href{https://github.com/nubot-nudt/gazebo_visual}{\tt gazebo\-\_\-visual}. However, the tutorial regarding compliation and etc. is still useful.

Please read the paper \href{https://www.trustie.net/organizations/23/publications}{\tt \char`\"{}\-Weijia Yao et al., A Simulation System Based on R\-O\-S and Gazebo for Robo\-Cup Middle Size League, 2015\char`\"{}} for more information.


\begin{DoxyItemize}
\item Maintainer status\-: maintained
\item Maintainer\-: Weijia Yao \href{mailto:abcgarden@126.com}{\tt abcgarden@126.\-com}
\item Author\-: Weijia Yao \href{mailto:abcgarden@126.com}{\tt abcgarden@126.\-com}
\item License\-: Apache
\item Bug / feature tracker\-: \href{https://github.com/nubot-nudt/single_nubot_gazebo/issues}{\tt https\-://github.\-com/nubot-\/nudt/single\-\_\-nubot\-\_\-gazebo/issues}
\item Source\-: git \href{https://github.com/nubot-nudt/single_nubot_gazebo}{\tt https\-://github.\-com/nubot-\/nudt/single\-\_\-nubot\-\_\-gazebo} (branch\-: master)
\end{DoxyItemize}

\section*{Recommended Operating Environment}


\begin{DoxyEnumerate}
\item Ubuntu 14.\-04;
\item R\-O\-S Indigo or R\-O\-S Jade. (It is recommended to install R\-O\-S Jade)
\item Gazebo 5.\-0 or above;
\item gazebo\-\_\-ros\-\_\-pkgs; (please read the {\bfseries N\-O\-T\-E} below for more information)
\item If you decide to use coach4sim with a G\-U\-I, you should make sure you have installed Qt5. The recommended install place is /opt. Other versions of Ubuntu, R\-O\-S or Gazebo may also work, but we have not tested yet.
\end{DoxyEnumerate}

{\bfseries N\-O\-T\-E\-:} Concerning how to install appropriate {\bfseries gazebo\-\_\-ros\-\_\-pkgs}, please read the following according to your own situation\-:
\begin{DoxyItemize}
\item 1. If you decide to use {\bfseries R\-O\-S Indigo}, please read the following\-: If you choose \char`\"{}desktop-\/full\char`\"{} install of R\-O\-S Indigo, there is a Gazebo 2.\-0 included initially. In order to install Gazebo 5.\-0/5.1, you should first remove Gazebo 2.\-0 by running\-: ({\bfseries The following command is dangerous; it might delete the whole R\-O\-S, so please do it carefully or you may find other ways to delete gazebo2}) {\ttfamily \$ sudo apt-\/get remove gazebo2$\ast$} Then you should be able to install Gazebo 5.\-0 now. To install gazebo\-\_\-ros\-\_\-pkgs compatible with Gazebo 5.\-0/5.1, run this command\-: {\ttfamily \$ sudo apt-\/get install ros-\/indigo-\/gazebo5-\/ros-\/pkgs ros-\/indigo-\/gazebo5-\/ros-\/control} H\-O\-W\-E\-V\-E\-R, if the above command does now work, these packages may be moved to other places. You can check out \href{https://github.com/ros-simulation/gazebo_ros_pkgs.git}{\tt gazebo\-\_\-ros} and download and install the correct version.
\item 2. If you decide to use {\bfseries R\-O\-S Jade} with {\bfseries gazebo 5.\-0 or 5.\-1}, read the following R\-O\-S Jade has gazebo\-\_\-ros\-\_\-pkgs with it; so you don't have to install gazebo\-\_\-ros\-\_\-pkgs again. However, you should do the following steps to fix some of the bugs in R\-O\-S Jade related to Gazebo\-:
\begin{DoxyItemize}
\item (a) {\ttfamily \$ sudo gedit /opt/ros/jade/lib/gazebo\-\_\-ros/gazebo} In this file, go to line 24 and delete the last '/'. So {\ttfamily setup\-\_\-path=\$(pkg-\/config -\/-\/variable=prefix gazebo)/share/gazebo/} is changed to {\ttfamily setup\-\_\-path=\$(pkg-\/config -\/-\/variable=prefix gazebo)/share/gazebo} You can read this link for more \href{http://answers.ros.org/question/215796/problem-for-install-gazebo_ros_package/}{\tt information}
\item (b) Install Gazebo 5. {\ttfamily \$ sudo apt-\/get install gazebo5} If this fails, try to run the \href{https://github.com/nubot-nudt/simatch/blob/master/gazebo5_install.sh}{\tt 'gazebo5\-\_\-install.\-sh'}(obtained from Gazebo's official website). Read for more \href{http://answers.ros.org/question/217970/ros-jade-and-gazebo-50-migration-problem/}{\tt information}
\item (c) Optional\-: copy resource files to the new gazebo folder. {\ttfamily \$ sudo cp -\/r /usr/share/gazebo-\/5.0/$\ast$ /usr/share/gazebo-\/5.1}
\end{DoxyItemize}
\item 3. If you decide to use {\bfseries R\-O\-S Jade} with {\bfseries gazebo 7.\-1}, read the following,
\begin{DoxyItemize}
\item (1) Install gazebo 7.\-0 by running \href{https://github.com/nubot-nudt/simatch/blob/master/gazebo7_install.sh}{\tt gazebo7\-\_\-install.\-sh}(obtained from Gazebo's official website);
\item (2) Then run this in the terminal\-:
\item {\ttfamily \$ sudo apt-\/get install ros-\/jade-\/gazebo7-\/ros-\/pkgs}
\end{DoxyItemize}
\end{DoxyItemize}

\section*{Complie}


\begin{DoxyEnumerate}
\item Go to the package root directory (single\-\_\-nubot\-\_\-gazebo)
\item If you already have C\-Make\-Lists.\-txt in the \char`\"{}src\char`\"{} folder, then you can skip this step. If not, run these commands\-:

``` \$ cd src \$ catkin\-\_\-init\-\_\-workspace \$ cd .. ```
\item \$ ./configure You may encounter errors related to Git. In this case, if you did not use Git, you could just ignore these errors.
\item \$ catkin\-\_\-make --pkg \hyperlink{namespacenubot__common}{nubot\-\_\-common}
\item \$ catkin\-\_\-make You may have to run catkin\-\_\-make for several times until 100\% success. If this does not work, please contact us.
\end{DoxyEnumerate}

\section*{Tutorials}

\subsection*{Part I. Overview}

The robot movement is realized by a Gazebo model plugin which is called \char`\"{}\-Nubot\-Gazebo\char`\"{} generated by source files \char`\"{}nubot\-\_\-gazebo.\-cc\char`\"{} and \char`\"{}nubot\-\_\-gazebo.\-hh\char`\"{}. Basically the essential part of the plugin is realizing basic motions\-: omnidirectional locomotion, ball-\/dribbling and ball-\/kicking.

Basically, this plugin subscribes to topic \char`\"{}nubotcontrol/velcmd\char`\"{} for omnidirecitonal movement and subscribes to service \char`\"{}\-Ball\-Handle\char`\"{} and \char`\"{}\-Shoot\char`\"{} for ball-\/dribbling and ball-\/kicking respectively. You can customize this code for your robot based on these messages and services as a convenient interface.

As for ball-\/dribbling, there are three ways for a robot to dribble a ball, i.\-e.

\begin{TabularC}{2}
\hline
\rowcolor{lightgray}\PBS\centering {\bf Method }&{\bf Description  }\\\cline{1-2}
\PBS\centering (a) Setting ball pose continually &This is the most accurate one; nubot would hardly lose control of the ball, but the visual effect is not very good (the ball does not rotate). \\\cline{1-2}
\PBS\centering (b) Setting ball secant velocity &This is less acurate than method (a) but more accurate than method (c). \\\cline{1-2}
\PBS\centering (c) Setting ball tangential velocity &This is the least accurate. If the robot moves fast, such as 3 m/s, it would probably lose control of the ball. However, this method achieves the best visual effect under low-\/speed condition. \\\cline{1-2}
\end{TabularC}
{\bfseries By default, we use method (c) for ball-\/dribbling.}

As for Gaussian noise, {\bfseries by default, Gaussian noise is N\-O\-T added}, but you can add it by changing the flag in \hyperlink{nubot__gazebo_8cc}{nubot\-\_\-gazebo.\-cc} in function update\-\_\-model\-\_\-info();

\subsection*{Part I\-I. Single robot automatic movement}

The robot will do motions according to states transfer graph. Steps are as follows\-:
\begin{DoxyEnumerate}
\item Go to the package root directory (single\-\_\-nubot\-\_\-gazebo)
\item source the setup.\-bash file\-: {\ttfamily \$ source devel/setup.\-bash}
\item {\ttfamily \$ roslaunch \hyperlink{namespacenubot__gazebo}{nubot\-\_\-gazebo} sdf\-\_\-nubot.\-launch}
\end{DoxyEnumerate}

\begin{quotation}
{\bfseries Note\-:} Every time you open a new terminal, you have to do step 2. You can also write this command into the $\sim$/.bashrc file so that you don't have to source it every time.

\end{quotation}


Finally, the robot rotates and translates with trajectory planning. That is, the robot accelerates at constant acceleration and stays at constant speed when it reaches the maximum velocity.

\subsection*{Part I\-I\-I. Keyboad control robot movement}


\begin{DoxyEnumerate}
\item In \hyperlink{nubot__gazebo_8cc}{nubot\-\_\-gazebo.\-cc}, comment \char`\"{}nubot\-\_\-auto\-\_\-control();\char`\"{} and uncomment \char`\"{}nubot\-\_\-be\-\_\-control();\char`\"{} in function Update\-Child().
\item Compile again and follow steps 1-\/3 listed in Part I\-I.
\item {\ttfamily \$ rosrun \hyperlink{namespacenubot__gazebo}{nubot\-\_\-gazebo} nubot\-\_\-teleop\-\_\-keyboard}
\end{DoxyEnumerate}

\subsection*{Part I\-V. Appendix}


\begin{DoxyEnumerate}
\item To launch an empty soccer field\-: {\ttfamily \$ roslaunch \hyperlink{namespacenubot__gazebo}{nubot\-\_\-gazebo} empty\-\_\-field.\-launch}
\item To launch the simulation world with rqt\-\_\-plot of nubot or ball's velocity\-: {\ttfamily \$ roslaunch \hyperlink{namespacenubot__gazebo}{nubot\-\_\-gazebo} sdf\-\_\-nubot.\-launch plot\-:=true}
\end{DoxyEnumerate}

\subsection*{Q\&A}